\section{Algorithm Analysis}

This section provides an overview and analysis of the important factors to consider when encoding polynomial systems of equations onto a quantum annealing architecture (resembling modern D-Wave hardware).

\subsection{Energy Gap}

The range of energies grows and shrinks exponentially with the bits of precision $n$, yielding a $\mathcal{O}(2^{-n})$ decay in the physical energy gap $\Delta'$ after the necessary rescaling operations.
When mixed-precision variables interact, $\Delta'$ is determined by the effective precision $\hat n$, which is the difference between the largest and smallest nonzero absolute value representable among all the variables in the system.
The number of additions $\alpha$ and number of multiplications $\mu$ have, respectively, linear and exponential effects on the maximum energy of the logical Hamiltonian.
Therefore, $\Delta'$ shrinks inverse linearly with respect to $\alpha$ and exponentially with respect to $\mu$.
Similarly as with addition, the number of equations $m$ in a system decreases $\Delta'$ at most inverse linearly since the equations do not directly interact, but may share some terms in common.
Combined, this yields a rate of decay for the physical energy gap (after rescaling) of
$$\Delta' \approx \mathcal{O}\bigg(\frac{1}{\alpha m 2^{\hat n \mu}}\bigg).$$
In general, it can be seen that $\Delta'$ will be primarily determined by the exponential decay $2^{-{\hat n}\mu}$.

\subsection{Number of Qubits}

Other than the number of qubits required to represent the binary numbers, some ancillary qubits may be introduced into the logical Hamiltonion during both quadratization and physical embedding.
Quadratization is performed analytically by the high-level tool described in Section 5, and so the ancillary qubit requirements due to quadratization can be easily analyzed.
When addition or subtraction is performed, no additional quadratization is required.
However, multiplication between two variables with $n_0$ and $n_1$ bits of precision requires $n_0 \times n_1$ ancillary qubits (one for each quadratization).
Given a maximum of $\mu$ multiplications per individual polynomial (single equation) and $n$ bits of precision, the number of ancillary qubits introduced through quadratization will be $\mathcal{O}(m n^\mu)$.
Notice that increasing the number of equations $m$ can at most linearly increase the number of ancillary quadratization terms required (when no variable products are shared between equations).
Also notice that the number of variables $d$ does not affect the number of ancillary qubits due to quadratization, for a fixed $\mu$.

When embedding the physical Hamiltonian for a polynomial system, two factors will dominate the number of necessary qubits.
First, the annealer must generate a fully connected subgraph of size $n^{(\mu + 1)/2}$.
Second, given some variable $\hat{\mathbf{x}}$ is multiplied by $k$ other distinct variables throughout a system, $\hat {\mathbf{x}}$ will have at least $k$ connections to each of its bits.
The modern Chimera graph structure is not conducive to fully-connected graphs nor small sets of qubits with high graph centrality, hence experimental results show a growth rate greater than $k n^\mu$ for large problems.

\subsection{Sensitivity to Perturbation}

Given the perturbed Hamiltonian model (\ref{eq:perturbed}), it is reasonable
to wonder how the perturbations $\delta_i$ and $\delta_{i,j}$ will affect
the quality of the solution.
In particular, one might wonder whether the effects could be great enough
so that the ground state of the perturbed problem is no longer the ground
state of the original problem.

Consider first an arbitrary physical Ising-Hamiltonian $\Tilde{\mathcal{H}}$ involving $N'$ physical qubits, as defined in (\ref{eq:physical}) and its corresponding perturbed Hamiltonian $\hat{\mathcal{H}}$ as in (\ref{eq:perturbed}).
Suppose that $|\delta_i|, |\delta_{i,j}| < \delta$, where $\delta>0$ is a constant that represents the largest magnitude perturbation possible on the current hardware.
Combining equations (\ref{eq:physical}) and (\ref{eq:perturbed}), applying the upper bounds $\delta$, and combining like terms yields
$$
\|\Tilde{\mathcal{H}} - \hat{\mathcal{H}}\|_\infty
= \sup_{\sigma} |\mathcal{H}(\sigma) - \hat{\mathcal{H}}(\sigma)|
\leq \sup_\sigma \left|\sum_{i'=1}^{N'} \delta \sigma_{i'} +
\sum_{i'=1}^{N'}\sum_{(i',j')\in G}\delta\sigma_{i'}\sigma_{j'} \right|.
$$

In the worst case, assuming nonzero connections between all physical qubits, the right hand side is upper bounded by $(N' + N' L')\delta$, where $L'$ is the number of connections per node in the physical graph $G$.
For the Chimera graph, each node has at most $6$ undirected edges.
So for the D-Wave Chimera architecture, the condition that guarantees a correct ground state for the physical Hamiltonian $\hat{\mathcal{H}}$ is $7N' \delta < 2\Delta'$.
