
%% General paragraph settings (no indent, big line skip).
\parindent=0pt
\parskip=9pt
\baselineskip=13pt plus 1pt minus 1pt \lineskiplimit=1pt

%% Red coloring.
\input colordvi
\def\BEGINC{\textRed}
\def\ENDC{\textBlack}
\def\red#1{\BEGINC#1\ENDC}
\let\beginred=\BEGINC
\let\endred=\ENDC

%% Fonts (for title)
\font\titlebf=cmbx20
\font\titlett=cmtt20
\font\ttVIII=cmtt8

%% EPS figures.
\input epsf

%% Title
{\titlebf Revision of {\titlett TOMS-2021-0028} \bigskip}

%% Body
Dear reviewers,

We are deeply grateful for the comments, questions, and references you
have all provided. In accordance with your feedback we have made
significant changes to the manuscript to improve clarity, expand the
summary of related literature, and incorporate visual comparisons with
existing spline software. We hope you find these changes have resulted
in an overall better manuscript.

Each numbered item and italicized text block below is a reviewer
comment (or summary of similar comments). Our responses immediately
follow.


\goodbreak
{\parindent=20pt \it

\item{(1)} The authors should include references to more related works
  and justify the differences between this work and the many prior
  works on $C^2$ interpolation. [given 23 references]

}

In order to properly satisfy this request, we have significantly
expanded the review of relevant research in the introduction (Section
1, paragraphs 3-6). This new content summarizes the individual tracks
represented by the conglomerate of references provided. After
reviewing all of those works we are still quite certain that this
manuscript represents a novel and meaningful contribution to the
literature, and have justified it's position among those works in the
revised manuscript. The most outstanding contribution of this work
remains the code, which applies sharp monotonicity constraints (unlike
any other $C^2$ methods) to ensure a monotone quintic interpolant
nearest to the initial nonmonotone (but otherwise {\it nice})
approximation is achievable.


\goodbreak
{\parindent=20pt \it

\item{(2)} The authors should include references to and comparisons
  with the following software package (which appears to be very
  similar).

\itemitem{} Costantini, P. An algorithm for computing shape-preserving
  interpolating splines of arbitrary degree, Journal of Computational
  and Applied Mathematics, 1998, 22(1), pp. 89-136.

\item{} While the concerns about subjectivity in comparisons are
  acknowledged, users will still want to know the context of this work
  and the trade-offs would they use this package instead of other
  spline algorithms. For this reason a comparison with other methods
  including visuals should be provided.

}

We appreciate this perspective and have included such comparisons in
the updated Section 5. The comparison shows methods from four
published spline software packages against MQSI for the same test
problems as originally in the paper (and associated test code). An
important point to note is that we were hesitant to include
comparisons with the published R package for Schumaker spline
interpolants, because (as is shown in the revised manuscript) the
package often produces incorrectly nonmonotone approximations to
data. In our opinion this undesirable outcome is a valid demonstration
of the numerical difficulties that arise when writing codes for spline
algorithms. They are not trivial to handle, and that underscores the
importance of thoroughly and carefully peer reviewed implementations
like what is provided by this work.


\goodbreak
{\parindent=20pt \it

\item{(3)} The monotonicity of Q(x) is not proved.

}

Per Section 1 paragraph 5, Section 2 paragraphs 3, 4 and 6, and
Algorithm 2 (all of the original submission) the theory proving the
sharp necessary and sufficient conditions for monotonicity is provided
by Ulrich and Watson [1994]. This code conforms precisely with the
tight boundary conditions referenced therein. The theory is quite
dense and it would seem inappropriate to reproduce all of that work in
this manuscript as well. To further clarify that the conditions for
monotonicity are proven, we have added an additional paragraph to
Section 5 explaining this fact.


\goodbreak
{\parindent=20pt \it

\item{(4)} The approximation order of the algorithm is not shown.

}

In the same paragraph of Section 5 referenced above, we have explained
that the approximation order of the method is that of any second order
approximation (because a local quadratic interpolant is used to
approximate initial derivative values), but also notes that the
asymptotic approximation order is often of little importance in the
context of (sparse) scattered data (the target application of most
shape preserving techniques, especially those achieving higher levels
of continuity).


\goodbreak
{\parindent=20pt \it

\item{(5)} Why use piecewise degree 5? In principle, piecewise degree 3
  is enough for $C^2$. The additional degrees of freedom for making
  the interpolant monotone could be obtained by adding knots between
  the breakpoints. It would be great if you could motivate your choice
  in more detail.

}

As stated in the comment, constructing polynomial pieces that
interpolate arbitrary function values and two derivatives at the ends
of an interval requires 6 degrees of freedom and hence an order 6
polynomial (degree 5, quintic) without adding additional knots. This
work builds on theory that does {\it not} incorporate new knots, and
it is our opinion that it is generally favorable to not insert new
knots. Without the insertion of additional knots, the ability to
construct lower order monotone $C^2$ splines is restricted to specific
arrangements of data. The restriction of having only a fixed set of
knots is somewhat arbitrary, but that is the application targeted by
this line of research.


\goodbreak
{\parindent=20pt \it

\item{(6)} Some statements depend on two variables being
  ``approximately equal'', but what does that mean? Equal up to some
  small threshold?

}

The definition of approximately equal here is when two floating point
numbers are within the machine epsilon (at their scale) of each other.
This is a rather standard approach for ensuring safe separation of
floating point numbers. We have added a sentence explaining this to
the pretexts of the algorithms that reference this operation.


\goodbreak
{\parindent=20pt \it
  
\item{(7)} In Algorithm 1 (and also in the text), you use the term
  ``curvature'' to refer to the second derivative. I would avoid this,
  since ``curvature'' is a term reserved for curves, but you are dealing
  with functions. Similarly, the term ``minimum curvature derivative''
  is inappropriate.

}

This has been rephrased throughout to reduce confusion.


\goodbreak
{\parindent=20pt \it
  
\item{(8)} In Algorithm 2, it might be better to say that f is a
  ``degree 5'' polynomial, instead of ``order 6''.

}

Noted and changed.

{\parindent=20pt \it
  
\item{(9)} Your binary search results in a particular set of
  derivatives, for which the local quintic piece is guaranteed to be
  monotone, but it is clearly not a unique choice. Alternatively, one
  could try to find, among all possible sets of derivatives and
  corresponding monotone quintic interpolants, the best, according to
  some criterion, e.g. least L2 norm. This would turn the problem into
  a constrained optimization problem, which is probably harder to
  solve, but it might, at the same time, give better results and
  overcome the limitation shown in Fig. 1. More motivating details
  would be appreciated.

}

This was a point of our own severe consideration before submitting the
manuscript and this comment reflected our original thoughts. In due
response to this we've implemented naive $L2$ minimization algorithm
over the second derivative that uses gradient descent over the spline
representation to enforce local monotonicity constraints in a
Lagrangian fashion (with an adaptive multiplier on monotonicity
violations). The results we achieved were quite mixed and we've
incorporated our commentary from this excursion into Section 5 of the
manuscript. In truth, formally expanding on this trajectory would
require an entirely new manuscript with deeper study of the problem,
while it is not immediately obvious that the results would be {\it
  better} than the present approach. The global $L^2$ minimizer of the
second derivative is simply the unique $C^2$ cubic spline interpolant,
but that is known to produce large and unnecessary local variations in
the approximation (along with not being shape preserving). Included in
this response (to reviewers only) are some of the preliminary visuals
we produced in exploring this approach. This is by no means final and
conclusive, but the results were mixed enough for us to warrant not
including them in the paper.


\centerline{\epsfxsize=4truein \epsffile{vis/comparisons/piecewise-polynomial/5-l2_minimizer.eps}}
\centerline{\epsfxsize=4truein \epsffile{vis/comparisons/large-tangent/5-l2_minimizer.eps}}
\centerline{\epsfxsize=4truein \epsffile{vis/comparisons/signal-decay/5-l2_minimizer.eps}}
\centerline{\epsfxsize=4truein \epsffile{vis/comparisons/random-monotone/5-l2_minimizer.eps}}
{\narrower\noindent {\bf $L^2$ Minimization Experiment }
{\ttVIII MQSI} (light gray) compared with an experimental global $L^2$
minimization approach (blue) on each of the test problems from the
paper. It is {\it very} difficult to consistently construct fits that
lack local oscillations as well as maintain monotonicity with this
approach. The method relies on the B-spline basis as well, which
affects the bias towards specific solutions when following the second
derivative $L^2$ minimizing gradient.  This work would require a very
significant line of research to properly resolve all these small
issues in a theoretically sound way.
\par}
\vfill \eject


\goodbreak
{\parindent=20pt \it
  
\item{(10)} It is quite probable that a user would want to convert the
  piecewise Hermite representation of the interpolant to quintic
  B-spline form. But the authors say in Sections 3 and 4 that they
  solve a banded linear system of length n (n the number of data
  points) to make this conversion. However, this is not necessary. At
  each data point, only three of the quintic B-splines are
  non-zero. Therefore, the three associated B-spline coefficients can
  be computed from the value, first, and second derivatives of the
  interpolant at that single point. Thus one can just solve a small
  3*3 linear system to get one group of 3 B-spline coefficients. Well,
  it requires a bit of algebraic work to write out this 3*3 linear
  system, but it can be done.

}

As mentioned in the manuscript Section 5 paragraph 4, the banded
linear system representation is favored for two reasons in this
package. The computational complexity order is unchanged by solving a
single banded system, while the global numerical stability is {\it
  theoretically} improved. The system of equations and number of
nonzero splines at each breakpoint is thoroughly documented in
{\ttVIII SPLINE.f90}, lines 172--206. In practice this approach is
almost certainly slower than solving individual linear systems, but
the increased global stability and decreased redundancy in
representation is favored here (since the banded linear system also
does not increase computational complexity). For users interested in
constructing alternative representations, the optional argument {\it
  \bf UV} documented in the code exists precisely for returning the
computed first and second derivatives at knots.


\goodbreak
{\parindent=20pt \it
  
\item{(11)} I'm confused by the O(1) and O(n) discussion in Section 4.
  If there are n data points then surely the main algorithm
  (estimating first and second derivatives at each data point) must
  require at least an O(n) algorithm, not O(1).

}

This was poorly worded on our part. The entire algorithm is O(n),
however the binary search itself has a fixed number of loop
iterations. The original intent of the statement was that the
complexity order of the outer loop (the binary search) is constant,
which ensures the overall complexity of O(n) is not changed. These
sentences in Section 4 paragraph 1 have been rephrased.


\goodbreak
{\parindent=20pt \it
  
\item{(12)} The output are values -- why not the spline coefficients? If
  the output are values one would hope for a visualization package.

}

The code includes suitable documentation and examples to conform to
any external visualization package desired by a user. Given the wide
variety of mechanisms by which a user might want to visualize or use
the results of this code, it is nearly impossible to provide a good
generic solution to visualization.

For the convenience of reviewers, we are providing the code with which
we generated the visuals for this manuscript. Although we note it has
very specific system requirements to operate effectively. Namely: (1)
running a POSIX shell (on macOS, Ubuntu, $\ldots$); (2) having
{\ttVIII python3} of at least version $3.7$ and the packages {\ttVIII
  numpy}, {\ttVIII tlux}, and {\ttVIII rpy2}; (3) having installed
{\ttVIII gfortran} (or otherwise reconfiguring the codes to use
another Fortran compiler of choice); and (4) having installed {\ttVIII
  R} (for Schumaker splines). The Fortran codes associated with this
work as well as those from TOMS 574 (quadratic $C^1$) and BVSPIS from
TOMS 770 are all wrapped to be callable directly from {\ttVIII
  python3}, which will be the parent process for all calls. A simple
test usage can be seen in the provided {\ttVIII python3} file {\ttVIII
  test.py}. To be abundantly clear, this code is {\bf not} intended
for review and publication, but rather convenient access to various
codes for reviewers to use for visualizing and comparing results.


\goodbreak
{\parindent=20pt \it
  
\item{(13)} I was not familiar with the term/notion ``quadratic facet
  model''. The term facet made me believe that the paper was on
  bivariate interpolation. Nothing in the abstract or title prevents
  this misconception.

}

The term ``univariate'' has been added to the abstract for greater
clarity on this point.

\vfill \eject

Throughout the manuscript, all modified text has been marked in
\red{red}. We hope that you all find these revisions to have improved
the manuscript, and we eagerly await more feedback.

Sincerely,

Thomas Lux, on behalf of all authors.

\bye


%% ----------------------------------------------------------------------
%%                        Useful TeX reference.
%% 
%% https://www.math.brown.edu/johsilve/ReferenceCards/TeXRefCard.v1.5.pdf
%% 
%% ----------------------------------------------------------------------
