\bigskip\hrule\bigskip\medskip
\heading{1. INTRODUCTION}

Many domains of science rely on smooth approximations to real-valued
functions over a closed interval. Piecewise polynomial functions
(splines) provide the smooth approximations for animation in graphics
[Herman et al.\ 2015; Quint 2003], aesthetic structural support in
architecture [Brennan 2020], efficient aerodynamic surfaces in
automotive and aerospace engineering [Brennan 2020], prolonged
effective operation of electric motors [Berglund et al.\ 2009], and
accurate nonparametric approximations in statistics [Knott
2012]. While polynomial interpolants and regressors apply broadly,
splines are often a good choice because they can approximate globally
complex functions while minimizing the local complexity of an
approximation.

It is often the case that the true underlying function or phenomenon
being modeled has known properties like convexity, positivity,
various levels of continuity, or monotonicity. Given a reasonable
amount of data, it quickly becomes difficult to achieve desirable
properties in a single polynomial function. In general, the
maintenance of function properties through interpolation/regression is
referred to as {\it shape preserving} [Fritsch and Carlson 1980;
Gregory 1985]. The specific properties the present algorithm will
preserve in approximations are monotonicity and $C^2$ continuity. In
addition to previously mentioned applications, these properties are
crucially important in statistics to the approximation of a cumulative
distribution function and subsequently the effective generation of
random numbers from a specified distribution [Ramsay 1988].  A spline
function with these properties could approximate a cumulative
distribution function to a high level of accuracy with relatively few
intervals. A twice continuously differentiable approximation to a
cumulative distribution function (CDF) would produce a corresponding
probability density function (PDF) that is continuously
differentiable, which is desirable.

The currently available software for monotone piecewise polynomial
interpolation includes quadratic [He and Shi 1998, McAllister and
Roulier 1981], cubic [Fritsch and Carlson 1980], and (with limited
application) quartic [Wang and Tan 2004; Piah and Unsworth 2011; Yao
and Nelson 2018] cases. In addition, a statistical method for
bootstrapping the construction of an arbitrarily smooth monotone fit
exists [Leitenstorfer and Tutz 2006], but the method does not take
advantage of known analytic properties of quintic polynomials.
Conversely the work on quadratic shape preserving splines by Schumaker
[1983] is widely utilized in theory for its conservation of both
monotonicity and convexity, but no mathematical software is published.

The code by Fritsch [1982] for $C^1$ cubic spline interpolation is the
predominantly utilized code for constructing monotone interpolants at
present. Theory has been provided for the quintic case [Ulrich and
Watson 1994; He{\ss} and Schmidt 1994] and that theory was recently
utilized in a proposed algorithm [Lux 2020] for monotone quintic
spline construction, however no published mathematical software
exists. The software presented here represents the first published
software package for producing $C^2$ shape preserving splines. This
work improves upon the algorithm presented by Lux et al. [2020] by
refactoring computations for improved numerical stability, estimating
minimum curvature derivatives at breakpoints with a quadratic facet
model, and using a binary search to reduce the magnitude of the
modifications made to initial derivative estimates when constructing a
monotone spline interpolant.

This work provides a Fortran 2003 subroutine {\tt MQSI} based on the
necessary and sufficient conditions in Ulrich and Watson [1994] for the
construction of monotone quintic spline interpolants of monotone data.
Precisely, the problem is, given a strictly increasing sequence $X_1<X_2<
\cdots <X_n$ of breakpoints with corresponding monotone increasing
function values $Y_1\le Y_2\le \cdots \le Y_n$, find a $C^2$ monotone
increasing quintic spline $Q(x)$ with the same breakpoints satisfying
$Q(X_i)=Y_i$ for $1\le i\le n$. ({\tt MQSI} actually does something slightly
more general, producing $Q(x)$ that is monotone increasing (decreasing)
wherever the data is monotone increasing (decreasing).)

The remainder of this paper is structured as follows: Section 2 provides
the algorithms for constructing a $C^2$ monotone quintic spline interpolant
to monotone data, Section 3 outlines the method of spline representation
($B$-spline basis) and evaluation, Section 4 analyzes the complexity and
sensitivity of the algorithms in {\tt MQSI}, and Section 5 presents 
timing data and some graphs of constructed interpolants.

