\heading{1. INTRODUCTION}

Many domains of science rely on smooth approximations to real-valued
functions over a closed interval. Piecewise polynomial functions
(splines) provide the smooth approximations for animation in graphics
[Herman et al.\ 2015; Quint 2003], aesthetic structural support in
architecture [Brennan 2020], efficient aerodynamic surfaces in
automotive and aerospace engineering [Brennan 2020], prolonged
effective operation of electric motors [Berglund et al.\ 2009], and
accurate nonparametric approximations in statistics [Knott
2012]. While polynomial interpolants and regressors apply broadly,
splines are often a good choice because they can approximate globally
complex functions while minimizing the local complexity of an
approximation.

It is often the case that the true underlying function or phenomenon
being modeled has known properties like convexity, positivity,
various levels of continuity, or monotonicity. Given a reasonable
amount of data, it quickly becomes difficult to achieve desirable
properties in a single polynomial function. In general, the
maintenance of function properties through interpolation/regression is
referred to as {\it shape preserving} [Fritsch and Carlson 1980;
Gregory 1985]. The specific properties the present algorithm will
preserve in approximations are monotonicity and $C^2$ continuity. In
addition to previously mentioned applications, these properties are
crucially important in statistics to the approximation of a cumulative
distribution function and subsequently the effective generation of
random numbers from a specified distribution [Ramsay 1988].  A spline
function with these properties could approximate a cumulative
distribution function to a high level of accuracy with relatively few
intervals. A twice continuously differentiable approximation to a
cumulative distribution function (CDF) would produce a corresponding
probability density function (PDF) that is continuously
differentiable, which is desirable.

\heading{1.1 Related Works}

[DeVore 1977] Proves approximation error bounds for monotone spline interpolation of data from monotone functions and accompanies the previous error bounds in the context of monotone polynomial interpolation of data from a monotone function.

[Beatson 1982] develops a quadratic spline curve fitting algorithm that maintains local convexity while also proving some error bounds. This method avoids inserting additional knots, but does not interpolate the data.

[Gregory and Delbourgo 1982] Proves necessary and sufficient conditions for a closed form solution to monotone C1 quadratic rational spline interpolation. Does not provide formal algorithm, nor code. Then again iterates towards a solution of nonlinear equations that ensure C2 continuity of quadratic rational splines. The solution they iterate towards exists and is unique. They provide proofs and methods for setting derivatives at the ends, no formal algorithms or code in [Delbourgo and Gregory 1983].

[Costantini 1986] Proves existence of splines that maintain monotonicity and convexity while having fixed continuity and order.

[Dougherty, Edelman, and Hyman 1989] Proposes an algorithm for constructing C2 Hermite monotone spline interpolants using proven sufficient, but not necessary conditions. As these bounds are not sharp, the space of allowable derivative and second derivative values at knots is unnecessarily shrunk. An algorithm is provided, but no code. Recommends against the direct optimization for smoothness parameters like L2, instead promoting problem specific definitions of 'geometric niceness'.

Fiorot and Tabka 1991] Proves a simple method for determining the existence of a monotone C2 cubic interpolating spline, but functions of this kind do not always exist for arbitrary convex monotone data sets.

[Delbourgo 1993] Continuous previous work by incorporating a cubic numerator, proves shape preservation when the correct value for a tension parameter is set but no method for setting that parameter. Accuracy depends on accurate derivative evaluations. No code nor formal algorithm is provided.

[Huynh 1993] Presents several nonlinear boundary conditions based on the median function to establish a fourth order monotonicity preserving constraints relying on divided difference initial derivative estimates. Overall generates a monotone C1 cubic interpolant to data.

[Pruess 1993] Proposes a cubic C2 shape preserving spline method with free parameters (mentions quintic splines are necessary for globally C2 monotone splines). Sufficient (but not necessary) conditions are provided for monotonicity.

[Manni 1997] Constructs a C2 cubic spline with two additional knots per interval. Then [Cravero and Manni 2003] Constructs C3 interpolating splines by progressively increasing tension through Bezier control polygons until shape parameters are satisfied, no code is provided.

[Abbas, Majid, and Ali 2012] No extra knots, rational functions, sufficiency and necessity proven, works for unequally spaced data, no code associated with the paper, no algorithm provided for the initialization for free shape parameters.

[Wang and Tan 2004] Propose a sufficient quartic C2 method with two free parameters.

[Piah and Unsworth 2011] improve upon the sufficient conditions of Wang and Tan for C2 quartic rational interpolation.

[Kvasov 2014] use sufficient conditions for monotonicity and convexity to form a globally weighted combination of C1 quadratic splines that satisfy shape constraints.

[Zhu and Han 2015] uses sufficient conditions over three free parameters to construct C2 rational quartic shape preserving splines. Finally [Han 2018] Proposes a method for making arbitrarily smooth shape preserving spline interpolants, but uses a centered quadratic to estimate derivatives and a sufficient set of bounding conditions to ensure shape preservation in practice. No necessary conditions are applied, no algorithm is provided, neither is any code provided.

[Yao and Nelson 2018] Develop novel sufficient conditions for monotone C2 quartic spline interpolation and maintain shape with added control points.

\heading{1.2 Available Software}

The currently available peer reviewed and published software for
monotone piecewise polynomial interpolation is severely lacking in
comparison with the number of published methodologies mentioned
above. This is indicative of how difficult it is to properly handle
the numerical conditions that arise when constructing and evaluating
precise shape preserving splines.

The piecewise quadratic $C^1$ method of [McAllister and Roulier 1981]
is available through TOMS, as well as the sufficient $C^2$ quintic
method of BVSPIS. The sufficient $C^1$ cubic spline method of [Fritsch and Carlson 1980],
is available through the SciPy Python package, and the $C^1$ rational
quadratic method of Schumaker [1983] is available as an R package.

cases. In addition, a statistical method for bootstrapping the
construction of an arbitrarily smooth monotone fit exists
[Leitenstorfer and Tutz 2006], but the method does not take advantage
of known analytic properties of quintic polynomials.  Conversely the
work on quadratic shape preserving splines by Schumaker [1983] is
widely utilized in theory for its conservation of both monotonicity
and convexity, but no mathematical software is published.

The code by Fritsch [1982] for $C^1$ cubic spline interpolation is the
predominantly utilized code for constructing monotone interpolants at
present. Theory has been provided for the quintic case [Ulrich and
Watson 1994; He{\ss} and Schmidt 1994] and that theory was recently
utilized in a proposed algorithm [Lux 2020] for monotone quintic
spline construction, however no published mathematical software
exists. The software presented here represents the first published
software package for producing $C^2$ shape preserving splines. This
work improves upon the algorithm presented by Lux et al. [2020] by
refactoring computations for improved numerical stability, estimating
minimum curvature derivatives at breakpoints with a quadratic facet
model, and using a binary search to reduce the magnitude of the
modifications made to initial derivative estimates when constructing a
monotone spline interpolant.

\heading{1.3 Overview}

This work provides a Fortran 2003 subroutine {\tt MQSI} based on the
necessary and sufficient conditions in Ulrich and Watson [1994] for the
construction of monotone quintic spline interpolants of monotone data.
Precisely, the problem is, given a strictly increasing sequence $X_1<X_2<
\cdots <X_n$ of breakpoints with corresponding monotone increasing
function values $Y_1\le Y_2\le \cdots \le Y_n$, find a $C^2$ monotone
increasing quintic spline $Q(x)$ with the same breakpoints satisfying
$Q(X_i)=Y_i$ for $1\le i\le n$. ({\tt MQSI} actually does something slightly
more general, producing $Q(x)$ that is monotone increasing (decreasing)
wherever the data is monotone increasing (decreasing).)

The remainder of this paper is structured as follows: Section 2 provides
the algorithms for constructing a $C^2$ monotone quintic spline interpolant
to monotone data, Section 3 outlines the method of spline representation
($B$-spline basis) and evaluation, Section 4 analyzes the complexity and
sensitivity of the algorithms in {\tt MQSI}, and Section 5 presents 
timing data and some graphs of constructed interpolants.

