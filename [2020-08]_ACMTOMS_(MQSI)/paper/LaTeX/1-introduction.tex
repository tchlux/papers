\section{INTRODUCTION}

Many domains of science rely on smooth approximations to real-valued
functions over a closed interval. Piecewise polynomial functions
(splines) provide the smooth approximations for animation in graphics
\cite{herman2015techniques,quint2003scalable}, aesthetic structural
support in architecture \cite{brennan2019measure}, efficient
aerodynamic surfaces in automotive and aerospace engineering
\cite{brennan2019measure}, prolonged effective operation of electric
motors \cite{berglund2009planning}, and accurate nonparametric
approximations in statistics \cite{knott2012interpolating}. While
polynomial interpolants and regressors apply broadly, splines are
often a good choice because they can approximate globally complex
functions while minimizing the local complexity of an approximation.

It is often the case that the true underlying function or phenomenon
being modeled has known properties like convexity, positivity, various
levels of continuity, or monotonicity. Given a reasonable amount of
data, it quickly becomes difficult to achieve desirable properties in
a single polynomial function. In general, the maintenance of function
properties through interpolation/regression is referred to as {\it
  shape preserving} \cite{fritsch1980monotone,gregory1985shape}. The
specific properties the present algorithm will preserve in
approximations are monotonicity and $C^2$ continuity.  Notably this
work does not consider convexity preservation, as Mulansky and Neamtu
\cite{mulansky1991existence} prove that there exists strictly convex
data sets which do not allow convex interpolation with $C^2$ splines
of fixed degree.  In addition to previously mentioned applications,
being $C^2$ and monotonicity preserving is crucially important in
statistics to the approximation of a cumulative distribution function
and subsequently the effective generation of random numbers from a
specified distribution \cite{ramsay1988monotone}.  A spline function
with these properties could approximate a cumulative distribution
function to a high level of accuracy with relatively few intervals. A
twice continuously differentiable approximation to a cumulative
distribution function (CDF) would produce a corresponding probability
density function (PDF) that is continuously differentiable, which is
desirable.

The existing research in shape preserving interpolatory splines is
rich and filled with many approaches for many different
applications. In the context of this work, the unique quality of shape
preserving spline algorithms can be observed through: (1) the type of
spline (rational or Hermite), (2) the polynomial order and level of
continuity, and (3) the sharpness, sufficiency, and necessity of the
conditions used to establish shape preserving properties. While the
choice of spline representation is less important, achieving higher
levels of continuity is often more difficult than lower levels of
continuity. Furthermore, all referenced algorithms establish
sufficient conditions for shape preservation, but it is less common
and much more difficult to provide {\it sharp and necessary}
conditions for monotonicity. The following two paragraphs highlight a
collection of research related to the construction of $C^1$ and $C^2$
shape preserving splines. From this foundation, the value and utility
of the code presented here can be more readily appreciated.

DeVore \cite{devore1977monotone} proves that monotone spline
interpolants have improved accuracy over their nonmonotone
counterparts when approximating monotone functions, while Costantini
\cite{constantini1986monotone} proves that monotone spline
interpolants to data of fixed continuity and order exist in
general. This groundwork establishes the potential for monotone spline
interpolation. Schumaker \cite{schumaker1983shape} and McAllister and
Roulier \cite{mcallister1981algorithm} (independently) establish
necessary and sufficient conditions for monotone $C^1$ quadratic
splines through the insertion of additional knots and Fritsch
\cite{fritsch1982LLNL} establishes simplified sufficient conditions
for monotone $C^1$ cubic splines that require no knot insertion.
Gregory and Delbourgo \cite{gregory1982piecewise} prove necessary and
sufficient conditions for a closed form solution to monotone $C^1$
quadratic rational spline interpolation with nonlinear boundary
equations and an iterative approach. Delbourgo and Gregory
\cite{delbourgo1983rational} extend that same work to achieve
sufficient conditions for $C^2$ continuity with a cubic rational
spline given additional tension parameters defined by users. Huynh
\cite{huynh1993accurate} similarly arrives at monotone $C^1$ cubic
spline interpolants and several necessary nonlinear boundary
conditions on monotonicity.

Continuing towards monotone $C^2$ spline interpolation, Fiorot and
Tabka \cite{fiorot1991shape} prove a simple method for determining the
existence of a monotone $C^2$ cubic interpolating spline, but note
that functions of this kind do not always exist for arbitrary convex
monotone data sets.  Pruess \cite{pruess1993shape} proposes sufficient
conditions for a cubic $C^2$ shape preserving spline method while
acknowledging that quintic splines are generally necessary for
monotone $C^2$ spline interpolants that do not insert new
knots. Similarly, Manni and Sablonni\`ere \cite{manni1997monotone}
construct sufficient conditions for a monotone $C^2$ cubic spline by
adding two additional knots per interval, then Cravero and Manni
\cite{cravero2003shape} extend that methodology to arrive at monotone
$C^3$ interpolating splines by progressively increasing tension
through Bezier control polygons. Costantini
\cite{constantini1997boundary} provides a method for monotone $C^2$
quintic spline interpolation based on sufficient and necessary
boundary value conditions, but the necessary conditions are not
sharp. Dougherty, et\ al.\ \cite{dougherty1989nonnegativity} construct
monotone $C^2$ Hermite splines and in commentary recommend against the
direct optimization for smoothness parameters like global $L^2$,
rather promoting problem specific definitions of geometric
niceness. Both Wang and Tan \cite{wang2004rational} as well as Yao and
Nelson \cite{yao2018unconditionally} propose sufficient conditions for
$C^2$ quartic splines, while Piah and Unsworth \cite{piah2011improved}
improve upon those sufficient conditions for $C^2$ quartic rational
spline interpolation. Abbas, et\ al.\ \cite{abbas2012monotonicity}
formulate a set of sufficient conditions for monotone $C^2$ cubic
rational splines and similar work is extended to $C^2$ quartic
rationals (and splines with arbitrary smoothness) by Zhu and Han
\cite{zhu2015rational,zhu2015shape}. Ulrich and Watson
\cite{ulrich1990positivity} arrive at sharp necessary and sufficient
conditions for monotone $C^2$ quintic splines, while He{\ss} and
Schmidt \cite{hess1994positive} produce similar sufficient conditions
for $C^2$ quintic splines. In addition, a method for constructing an
arbitrarily smooth monotone fit exists
\cite{leitenstorfer2006generalized} as well as a method for
arbitrarily smooth shape preservation \cite{han2018shape}, but these
methods are enabled by sufficient conditions only.

The currently available peer reviewed and published software for
monotone piecewise polynomial interpolation is severely limited in
comparison with the number of published approaches mentioned
above. This is partially indicative of how difficult it is to properly
handle the numerical conditions that arise when constructing and
evaluating precise shape preserving splines, as well as how difficult
it is to create code that correctly conforms with the theoretical
expectations. The sufficient $C^1$ cubic spline method PCHIP of
Fritsch and Carlson \cite{fritsch1980monotone} is available through
the SciPy Python package ($>$1 billion downloads), and the $C^1$
piecewise quadratic method of Schumaker \cite{schumaker1983shape} is
available as an R package (300,000 downloads). The $C^1$ piecewise
quadratic method of McAllister and Roulier
\cite{mcallister1981algorithm,mcallister1981shape} is available in
FORTRAN (1000 downloads) and the sufficient $C^2$ piecewise quintic
method of Costantini
\cite{constantini1997boundary,constantini1997bvspis} BVSPIS in FORTRAN
as well (500 downloads). Based on publicly available download records,
it is assumed that the code by Fritsch \cite{fritsch1980monotone} for
monotone $C^1$ cubic spline interpolation is the predominant code for
constructing monotone interpolants at present.

The theory for {\it sharp} necessary and sufficient bounds for
monotone $C^2$ quintic interpolation has been provided
\cite{ulrich1990positivity,ulrich1994positivity} and that theory was
recently utilized in a proposed algorithm
\cite{lux2020algorithm,lux2020thesis} for monotone $C^2$ quintic
spline construction, however no published mathematical software exists
for the quintic case based on sharp monotonicity conditions. The
software presented here represents the first published software
package for producing $C^2$ shape preserving splines based on sharp
monotonicity conditions. This work improves upon the algorithm
presented by Lux et\ al.\ \cite{lux2020algorithm} by refactoring
computations for improved numerical stability, estimating minimum
magnitude second derivatives at breakpoints with a quadratic facet
model, and using a binary search to reduce the magnitude of the
modifications made to initial derivative estimates when constructing a
monotone spline interpolant.

\subsection*{Overview}
This work provides a Fortran 2003 subroutine {\tt MQSI} based on the
sharp necessary and sufficient conditions in Ulrich and Watson
\cite{ulrich1994positivity} for the construction of monotone $C^2$
quintic spline interpolants of monotone data. Precisely, the problem
is, given a strictly increasing sequence $X_1<X_2< \cdots <X_n$ of
breakpoints with corresponding monotone increasing function values
$Y_1\le Y_2\le \cdots \le Y_n$, find a $C^2$ monotone increasing
quintic spline $Q(x)$ with the same breakpoints satisfying
$Q(X_i)=Y_i$ for $1\le i\le n$. ({\tt MQSI} actually does something
slightly more general, producing $Q(x)$ that is monotone increasing
(decreasing) wherever the data is monotone increasing (decreasing).)

The remainder of this paper is structured as follows: Section 2
provides the algorithms for constructing a monotone $C^2$ quintic
spline interpolant to monotone data, Section 3 outlines the method of
spline representation ($B$-spline basis) and evaluation, Section 4
analyzes the complexity and sensitivity of the algorithms in {\tt
  MQSI}, and Section 5 presents timing data and some graphs of
constructed interpolants as well as a visual comparison with existing
monotone spline packages.

