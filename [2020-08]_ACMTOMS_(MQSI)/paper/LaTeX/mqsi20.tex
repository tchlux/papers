%%
%% This is file `sample-manuscript.tex',
%% generated with the docstrip utility.
%%
%% The original source files were:
%%
%% samples.dtx  (with options: `manuscript')
%% 
%% IMPORTANT NOTICE:
%% 
%% For the copyright see the source file.
%% 
%% Any modified versions of this file must be renamed
%% with new filenames distinct from sample-manuscript.tex.
%% 
%% For distribution of the original source see the terms
%% for copying and modification in the file samples.dtx.
%% 
%% This generated file may be distributed as long as the
%% original source files, as listed above, are part of the
%% same distribution. (The sources need not necessarily be
%% in the same archive or directory.)
%%
%%
%% Commands for TeXCount
%TC:macro \cite [option:text,text]
%TC:macro \citep [option:text,text]
%TC:macro \citet [option:text,text]
%TC:envir table 0 1
%TC:envir table* 0 1
%TC:envir tabular [ignore] word
%TC:envir displaymath 0 word
%TC:envir math 0 word
%TC:envir comment 0 0
%%
%%
%% The first command in your LaTeX source must be the \documentclass
%% command.
%%
%% For submission and review of your manuscript please change the
%% command to \documentclass[manuscript, screen, review]{acmart}.
%%
%% When submitting camera ready or to TAPS, please change the command
%% to \documentclass[sigconf]{acmart} or whichever template is required
%% for your publication.
%%
%%
%% \documentclass[manuscript,screen,review]{acmart}
\documentclass[manuscript]{acmart}


%% Graphics package.
\usepackage{graphicx}

%% Rights management information.  This information is sent to you
%% when you complete the rights form.  These commands have SAMPLE
%% values in them; it is your responsibility as an author to replace
%% the commands and values with those provided to you when you
%% complete the rights form.
\setcopyright{acmcopyright}
\acmJournal{TOMS}
\acmYear{2022} \acmVolume{1} \acmNumber{1} \acmArticle{1} \acmMonth{1} \acmPrice{15.00}\acmDOI{10.1145/3570157}
\acmDOI{10.1145/3570157}
%% \copyrightyear{2022}

%% %% These commands are for a PROCEEDINGS abstract or paper.
%% \acmConference[Conference acronym 'XX]{Make sure to enter the correct
%%   conference title from your rights confirmation emai}{June 03--05,
%%   2018}{Woodstock, NY}
%%
%%  Uncomment \acmBooktitle if the title of the proceedings is different
%%  from ``Proceedings of ...''!
%%
\acmBooktitle{TOMS '22: ACM Transactions on Mathematical Software,
 October 01--10, 2022, Blacksburg, VA}
\acmPrice{0.00}
\acmISBN{978-1-4503-XXXX-X/22/10}


%% Find references with:
%%   frex '[\a][[\s\w.\][[\s\w.\]?[[\s\w.\]?[[\s\w.\]?[[\s\w.\]?[[\s\w.\]?[[\s\w.\]?[[\s\w.\]?[[\s\w.\]?[[\s\w.\]?[[\s\w.\]?[[\s\w.\]?[[\s\w.\]?[[\s\w.\]?[[\s\w.\]?[[\s\w.\]?[[\s\w.\]?[[\s\w.\]?[[\s\w.\]?[[\s\w.\]?[[\s\w.\]?[[\s\w.\]?[[\s\w.\]?[[\s\w.\]?[[\s\w.\]?[[\s\w.\]?[[\s\w.\]?[[\s\w.\]?[[\s\w.\]?[[\s\w.\]?[[\s\w.\]?[\d][\d][\d][\d]' '[.]tex{.}'

%%
%% Submission ID.
%% Use this when submitting an article to a sponsored event. You'll
%% receive a unique submission ID from the organizers
%% of the event, and this ID should be used as the parameter to this command.
%%\acmSubmissionID{123-A56-BU3}

%%
%% For managing citations, it is recommended to use bibliography
%% files in BibTeX format.
%%
%% You can then either use BibTeX with the ACM-Reference-Format style,
%% or BibLaTeX with the acmnumeric or acmauthoryear sytles, that include
%% support for advanced citation of software artefact from the
%% biblatex-software package, also separately available on CTAN.
%%
%% Look at the sample-*-biblatex.tex files for templates showcasing
%% the biblatex styles.
%%

%%
%% The majority of ACM publications use numbered citations and
%% references.  The command \citestyle{authoryear} switches to the
%% "author year" style.
%%
%% If you are preparing content for an event
%% sponsored by ACM SIGGRAPH, you must use the "author year" style of
%% citations and references.
%% Uncommenting the next command will enable that style.
%% \citestyle{acmauthoryear}

%%
%% end of the preamble, start of the body of the document source.
\begin{document}

%%
%% The "title" command has an optional parameter,
%% allowing the author to define a "short title" to be used in page headers.
\title{Algorithm XXXX: MQSI---Monotone Quintic Spline Interpolation}

%%
%% The "author" command and its associated commands are used to define
%% the authors and their affiliations.
%% Of note is the shared affiliation of the first two authors, and the
%% "authornote" and "authornotemark" commands
%% used to denote shared contribution to the research.
\author{Thomas C. H. Lux}
\email{tchlux@vt.edu}
\orcid{0000-0002-1858-4724}
\author{Layne T. Watson}
\email{ltwatson@computer.org}
\orcid{0000-0003-2009-107X}
\affiliation{%
  \institution{Virginia Polytechnic Institute and State University}
  \streetaddress{Departments of Computer Science, Mathematics, and Aerospace and Ocean Engineering}
  \city{Blacksburg}
  \state{Virginia}
  \postcode{24061}
  \country{USA}
}

\author{Tyler H. Chang}
\email{thchang@vt.edu}
\orcid{0000-0001-9541-7041}
\affiliation{%
  \institution{Argonne National Laboratory}
  \streetaddress{9700 South Cass Avenue, Bldg. 240}
  \city{Lemont}
  \state{Illinois}
  \postcode{60439}
  \country{USA}
}

\author{William I. Thacker}
\email{thackerw@winthrop.edu}
\orcid{0000-0001-7806-6593}
\affiliation{%
  \institution{Winthrop University}
  \city{Rock Hill}
  \state{South Carolina}
  \postcode{29733}
  \country{USA}
}

%%
%% By default, the full list of authors will be used in the page
%% headers. Often, this list is too long, and will overlap
%% other information printed in the page headers. This command allows
%% the author to define a more concise list
%% of authors' names for this purpose.
\renewcommand{\shortauthors}{Lux et al.}

%%
%% The abstract is a short summary of the work to be presented in the
%% article.
\begin{abstract}
  MQSI is a Fortran 2003 subroutine for constructing monotone quintic
  spline interpolants to univariate monotone data. Using sharp
  theoretical monotonicity constraints, first and second derivative
  estimates at data provided by a quadratic facet model are refined to
  produce a univariate $\scriptstyle C^2$ monotone interpolant.
  Algorithm and implementation details, complexity and sensitivity
  analyses, usage information, a brief performance study, and
  comparisons with other spline approaches are included.
\end{abstract}

%%
%% The code below is generated by the tool at http://dl.acm.org/ccs.cfm.
%% Please copy and paste the code instead of the example below.
%%
\begin{CCSXML}
<ccs2012>
<concept>
<concept_id>10002950.10003714.10003715.10003722</concept_id>
<concept_desc>Mathematics of computing~Interpolation</concept_desc>
<concept_significance>500</concept_significance>
</concept>
<concept>
<concept_id>10002950.10003714.10003715.10003720</concept_id>
<concept_desc>Mathematics of computing~Computations on polynomials</concept_desc>
<concept_significance>500</concept_significance>
</concept>
<concept>
<concept_id>10002950.10003705.10011686</concept_id>
<concept_desc>Mathematics of computing~Mathematical software performance</concept_desc>
<concept_significance>500</concept_significance>
</concept>
<concept>
<concept_id>10010405.10010432.10010442</concept_id>
<concept_desc>Applied computing~Mathematics and statistics</concept_desc>
<concept_significance>300</concept_significance>
</concept>
<concept>
<concept_id>10002950.10003714.10003740</concept_id>
<concept_desc>Mathematics of computing~Quadrature</concept_desc>
<concept_significance>300</concept_significance>
</concept>
<concept>
<concept_id>10002950.10003648.10003703</concept_id>
<concept_desc>Mathematics of computing~Distribution functions</concept_desc>
<concept_significance>100</concept_significance>
</concept>
<concept>
<concept_id>10010147.10010371.10010396.10010399</concept_id>
<concept_desc>Computing methodologies~Parametric curve and surface models</concept_desc>
<concept_significance>100</concept_significance>
</concept>
</ccs2012>
\end{CCSXML}

\ccsdesc[500]{Mathematics of computing~Interpolation}
\ccsdesc[500]{Mathematics of computing~Computations on polynomials}
\ccsdesc[500]{Mathematics of computing~Mathematical software performance}
\ccsdesc[300]{Applied computing~Mathematics and statistics}
\ccsdesc[300]{Mathematics of computing~Quadrature}
\ccsdesc[100]{Mathematics of computing~Distribution functions}
\ccsdesc[100]{Computing methodologies~Parametric curve and surface models}
%%
%% Keywords. The author(s) should pick words that accurately describe
%% the work being presented. Separate the keywords with commas.
\keywords{Software, Quintic spline, Interpolation, B-spline, Univariate, Shape preserving}

\received{28 August 2020}
\received[revised]{29 March 2021}
\received[revised]{28 March 2022}
\received[accepted]{14 September 2022}

%%
%% This command processes the author and affiliation and title
%% information and builds the first part of the formatted document.
\maketitle

%% Define some custom commands.
\def\thickrule{\hrule\hrule\hrule}
\def\codent{\hskip 1.0em} % <- 'tab' for indentation.

% fonts
\font\itVIII=cmti8 \font\slVIII=cmsl8 \font\bfVIII=cmbx8
\font\ttVIII=cmtt8 \font\caps=cmcsc10 \font\capsVIII=cmcsc10 at 8pt
\font\ttXVIII=cmtt12 at 18pt \font\eightmi=cmmi8 \font\eightsy=cmsy8
\font\rmeight=cmr8  \def\rmVIII{\rmeight \baselineskip=10pt plus 2pt
minus 1pt \lineskiplimit=2pt \lineskip=2pt plus 1pt
\parskip=0pt \textfont0=\rmeight \textfont1=\eightmi \textfont2=\eightsy}
\font\hlvXVIII=phvr at 18pt \font\hlvVIII=phvr at 8pt \font\hlv=phvr
% math definition
\def\:{\mathrel{:\mathord=}}

%% Section 1
\input 1-introduction

%% Section 2
\input 2-monotone-quintic

%% Section 3
\input 3-spline-rep

%% Section 4
\input 4-analysis

%% Section 5
\input 5-applications


%% \section{Figures}

%% The ``\verb|figure|'' environment should be used for figures. One or
%% more images can be placed within a figure. If your figure contains
%% third-party material, you must clearly identify it as such, as shown
%% in the example below.
%% \begin{figure}[h]
%%   \centering
%%   \includegraphics[width=\linewidth]{sample-franklin}
%%   \caption{1907 Franklin Model D roadster. Photograph by Harris \&
%%     Ewing, Inc. [Public domain], via Wikimedia
%%     Commons. (\url{https://goo.gl/VLCRBB}).}
%%   \Description{A woman and a girl in white dresses sit in an open car.}
%% \end{figure}

%% Your figures should contain a caption which describes the figure to
%% the reader.

%% Figure captions are placed {\itshape below} the figure.

%% Every figure should also have a figure description unless it is purely
%% decorative. These descriptions convey what’s in the image to someone
%% who cannot see it. They are also used by search engine crawlers for
%% indexing images, and when images cannot be loaded.

%% A figure description must be unformatted plain text less than 2000
%% characters long (including spaces).  {\bfseries Figure descriptions
%%   should not repeat the figure caption – their purpose is to capture
%%   important information that is not already provided in the caption or
%%   the main text of the paper.} For figures that convey important and
%% complex new information, a short text description may not be
%% adequate. More complex alternative descriptions can be placed in an
%% appendix and referenced in a short figure description. For example,
%% provide a data table capturing the information in a bar chart, or a
%% structured list representing a graph.  For additional information
%% regarding how best to write figure descriptions and why doing this is
%% so important, please see
%% \url{https://www.acm.org/publications/taps/describing-figures/}.

%% \subsection{The ``Teaser Figure''}

%% A ``teaser figure'' is an image, or set of images in one figure, that
%% are placed after all author and affiliation information, and before
%% the body of the article, spanning the page. If you wish to have such a
%% figure in your article, place the command immediately before the
%% \verb|\maketitle| command:
%% \begin{verbatim}
%%   \begin{teaserfigure}
%%     \includegraphics[width=\textwidth]{sampleteaser}
%%     \caption{figure caption}
%%     \Description{figure description}
%%   \end{teaserfigure}
%% \end{verbatim}

%% \section{Citations and Bibliographies}

%% The use of \BibTeX\ for the preparation and formatting of one's
%% references is strongly recommended. Authors' names should be complete
%% --- use full first names (``Donald E. Knuth'') not initials
%% (``D. E. Knuth'') --- and the salient identifying features of a
%% reference should be included: title, year, volume, number, pages,
%% article DOI, etc.

%% The bibliography is included in your source document with these two
%% commands, placed just before the \verb|\end{document}| command:
%% \begin{verbatim}
%%   \bibliographystyle{ACM-Reference-Format}
%%   \bibliography{bibfile}
%% \end{verbatim}
%% where ``\verb|bibfile|'' is the name, without the ``\verb|.bib|''
%% suffix, of the \BibTeX\ file.

%% Citations and references are numbered by default. A small number of
%% ACM publications have citations and references formatted in the
%% ``author year'' style; for these exceptions, please include this
%% command in the {\bfseries preamble} (before the command
%% ``\verb|\begin{document}|'') of your \LaTeX\ source:
%% \begin{verbatim}
%%   \citestyle{acmauthoryear}
%% \end{verbatim}


%%   Some examples.  A paginated journal article \cite{Abril07}, an
%%   enumerated journal article \cite{Cohen07}, a reference to an entire
%%   issue \cite{JCohen96}, a monograph (whole book) \cite{Kosiur01}, a
%%   monograph/whole book in a series (see 2a in spec. document)
%%   \cite{Harel79}, a divisible-book such as an anthology or compilation
%%   \cite{Editor00} followed by the same example, however we only output
%%   the series if the volume number is given \cite{Editor00a} (so
%%   Editor00a's series should NOT be present since it has no vol. no.),
%%   a chapter in a divisible book \cite{Spector90}, a chapter in a
%%   divisible book in a series \cite{Douglass98}, a multi-volume work as
%%   book \cite{Knuth97}, a couple of articles in a proceedings (of a
%%   conference, symposium, workshop for example) (paginated proceedings
%%   article) \cite{Andler79, Hagerup1993}, a proceedings article with
%%   all possible elements \cite{Smith10}, an example of an enumerated
%%   proceedings article \cite{VanGundy07}, an informally published work
%%   \cite{Harel78}, a couple of preprints \cite{Bornmann2019,
%%     AnzarootPBM14}, a doctoral dissertation \cite{Clarkson85}, a
%%   master's thesis: \cite{anisi03}, an online document / world wide web
%%   resource \cite{Thornburg01, Ablamowicz07, Poker06}, a video game
%%   (Case 1) \cite{Obama08} and (Case 2) \cite{Novak03} and \cite{Lee05}
%%   and (Case 3) a patent \cite{JoeScientist001}, work accepted for
%%   publication \cite{rous08}, 'YYYYb'-test for prolific author
%%   \cite{SaeediMEJ10} and \cite{SaeediJETC10}. Other cites might
%%   contain 'duplicate' DOI and URLs (some SIAM articles)
%%   \cite{Kirschmer:2010:AEI:1958016.1958018}. Boris / Barbara Beeton:
%%   multi-volume works as books \cite{MR781536} and \cite{MR781537}. A
%%   couple of citations with DOIs:
%%   \cite{2004:ITE:1009386.1010128,Kirschmer:2010:AEI:1958016.1958018}. Online
%%   citations: \cite{TUGInstmem, Thornburg01, CTANacmart}.
%%   Artifacts: \cite{R} and \cite{UMassCitations}.

%%
%% The acknowledgments section is defined using the "acks" environment
%% (and NOT an unnumbered section). This ensures the proper
%% identification of the section in the article metadata, and the
%% consistent spelling of the heading.
\begin{acks}
This work was supported by National Science Foundation Grants
CNS-1565314, CNS-1838271, and DGE-154362.
\end{acks}

%%
%% The next two lines define the bibliography style to be used, and
%% the bibliography file.
\bibliographystyle{ACM-Reference-Format}
\bibliography{mqsi20}
%% \printbibliography

\end{document}
\endinput
%%
%% End of file `sample-manuscript.tex'.
