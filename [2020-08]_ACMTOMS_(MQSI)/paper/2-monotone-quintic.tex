
\heading{2. MONOTONE QUINTIC INTERPOLATION}

In order to construct a monotone quintic interpolating spline, two
primary problems must be solved. First, reasonable derivative values
at data points need to be estimated. Second, the estimated derivative
values need to be modified to enforce monotonicity on all polynomial
pieces.

Fritsch and Carlson [1980] originally proposed the use of central
differences to estimate derivatives, however this often leads to extra
and unnecessary {\it wiggles} in the spline when used to approximate
second derivatives. Modern shape-preserving spline implementations use
a weighted harmonic mean to estimate derivative values at breakpoints
[Moler 2008], however this method also yields approximations whose
second derivative functions often have large \red{local} $L^2$ norm
(approximations with large {\it wiggle}). In an attempt to capture the
local shape of the data while minimizing wiggle, this package uses a
facet model from image processing [Haralick and Watson 1981] to
estimate first and second derivatives at breakpoints. Rather than
picking a local linear or quadratic fit with minimal residual, this
work uses a quadratic facet model that selects the local quadratic
interpolant with minimum magnitude \red{second derivative}.

%% ===================================================================
%%                   Algorithm 1: QUADRATIC_FACET
\vskip 5mm
{\parindent 0mm
{\bf Algorithm 1:}
{\tt QUADRATIC\_FACET}$\bigl(X(1{:}n)$, $Y(1{:}n)$, $i \bigr)$

\nobreak
where $X_j$, $Y_j$ $\in \bbb{R}$ for $j = 1$, $\ldots$, $n$, $1 \le i
\le n$, and $n \ge 3$. Returns the \red{first and second derivative} at 
$X_i$ of the local quadratic interpolant with minimum magnitude
second derivative. \red{Approximate equality is denoted with
$\approx$ and considers two numbers to be equal once they are within
the machine precision $\epsilon$ of each other.}

}
{\parindent=3mm
%% -------------------------------------------------------------------
\item{} {\tt if} $\bigl((i\ne1\wedge Y_i \approx Y_{i-1})$ {\tt or}
$(i\ne n\wedge Y_i \approx Y_{i+1})\bigr)$ {\tt then return $(0,0)$}

\item{} {\tt else if} $i=1$ {\tt then}
\itemitem{} $f_1\:\hbox{interpolant to }(X_1,Y_1)$,
  $(X_2,Y_2)$, and $(X_3,Y_3)$.
\itemitem{} {\tt if} $\bigl(Df_1(X_1)(Y_2-Y_1)<0\bigr)$ {\tt then return}
  $\bigl(0,0\bigr)$
\itemitem{} {\tt else return} $\bigl(Df_1(X_1),D^2f_1\bigr)$
\itemitem{} {\tt endif}

\item{} {\tt else if} $i=n$ {\tt then}
\itemitem{} $f_1\:\hbox{interpolant to }(X_{n-2},Y_{n-2})$,
  $(X_{n-1},Y_{n-1})$, and $(X_n,Y_n)$.
\itemitem{} {\tt if} $\bigl(Df_1(X_n)(Y_n-Y_{n-1})<0\bigr)$ {\tt then return}
  $\bigl(0,0\bigr)$
\itemitem{} {\tt else return} $\bigl(Df_1(X_n),D^2f_1\bigr)$
\itemitem{} {\tt endif}

\item{} {\tt else if} $\bigl(1<i<n\wedge (Y_{i+1} - Y_i)(Y_i - Y_{i-1}) < 0
\bigr)$ {\tt then}
\itemitem{} The point $(X_i$, $Y_i)$ is an extreme point. The
quadratic with minimum magnitude \red{second derivative} that has slope
zero at $X_i$ will be the facet chosen.
\itemitem{} $f_1\:\hbox{interpolant to }(X_{i-1},Y_{i-1})$, $(X_i,Y_i)$,
  and $Df_1(X_i) = 0$.
\itemitem{} $f_2\:\hbox{interpolant to }(X_i,Y_i)$, $(X_{i+1},Y_{i+1})$,
  and $Df_2(X_i) = 0$.
\itemitem{} {\tt if} $\bigl(|D^2f_1| \leq |D^2f_2|\bigr)$ {\tt then
  return} $\bigl(0$, $D^2f_1\bigr)$
\itemitem{} {\tt else return} $\bigl(0$, $D^2f_2\bigr)$
\itemitem{} {\tt endif}

\item{} {\tt else}
\itemitem{} {The point $(X_i$, $Y_i)$ is in a monotone segment of
data. In the following, it is possible that $f_1$ or $f_3$ do
not exist because $i \in \{2, n-1\}$. In those cases, the minimum
magnitude \red{second derivative} among existing quadratics is chosen.}
\itemitem{} $f_1\:\hbox{interpolant to }(X_{i-2},Y_{i-2})$,
  $(X_{i-1},Y_{i-1})$, and $(X_i,Y_i)$.
\itemitem{} $f_2\:\hbox{interpolant to }(X_{i-1},Y_{i-1})$,
  $(X_i,Y_i)$, and $(X_{i+1},Y_{i+1})$.
\itemitem{} $f_3\:\hbox{interpolant to }(X_i,Y_i)$,
  $(X_{i+1},Y_{i+1})$, and $(X_{i+2},Y_{i+2})$.

\itemitem{} {\tt if} $\bigl(Df_1(X_i)(Y_i-Y_{i-1}) \ge 0 \wedge
  |D^2f_1| = \min\bigl\{ |D^2f_1|,
  |D^2f_2|, |D^2f_3|\bigr\} \bigr)$ {\tt then}
\itemitem{} \codent {\tt return} $\bigl(Df_1(X_i),D^2f_1\bigr)$

\itemitem{} {\tt else if} $\bigl(Df_2(X_i)(Y_i-Y_{i-1}) \ge 0 \wedge
  |D^2f_2| = \min\bigl\{ |D^2f_1|,
  |D^2f_2|, |D^2f_3|\bigr\} \bigr)$
\itemitem{} \codent {\tt then return} $\bigl(Df_2(X_i),D^2f_2\bigr)$

\itemitem{} {\tt else if} $\bigl(Df_3(X_i)(Y_{i+1}-Y_i) \ge 0\bigr)$ {\tt then}
\itemitem{} \codent {\tt return} $\bigl(Df_3(X_i),D^2f_3\bigr)$
\itemitem{} {\tt else return} $(0, 0)$
\itemitem{} {\tt endif}
\item{} {\tt endif}
}
\vskip 5mm
%% ----------------------------------------------------------------------

For constructing a quadratic interpolant in $x$ over the interval
$[L,R]$, the Chebyshev basis $1$, $z$, $2z^2-1$ is used, where
$z={x-(L+R)/2 \over (R-L)/2}$. The estimated derivative values by the
quadratic facet model are not guaranteed to produce monotone quintic
polynomial segments. Ulrich and Watson [1990] established tight
constraints on the monotonicity of a quintic polynomial piece and
He{\ss} and Schmidt [1994] gave simpler conditions for a special case.
The following algorithm implements a sharp check for monotonicity by
considering the nondecreasing case. The nonincreasing case is handled
similarly.

%% ===================================================================
%%                    Algorithm 2: IS_MONOTONE
\vskip 5mm
{\parindent 0mm
{\bf Algorithm 2:}
{\tt IS\_MONOTONE}$\bigl(x_0$, $x_1$, $f(x_0)$, $Df(x_0)$,
$D^2f(x_0)$, $f(x_1)$, $Df(x_1)$, $D^2f(x_1) \bigr)$

\nobreak

where $x_0$, $x_1 \in {\bbb R}$, $x_0 < x_1$, and $f$ is an order six
polynomial defined by $f(x_0)$, $Df(x_0)$, $D^2f(x_0)$, $f(x_1)$,
$Df(x_1)$, $D^2f(x_1)$. Returns {\tt TRUE} if $f$ is monotone
increasing on $[x_0,x_1]$. \red{Approximate equality is denoted with
$\approx$ and considers two numbers to be equal once they are within
the machine precision $\epsilon$ of each other.}

}
{\parindent=6mm
%% -------------------------------------------------------------------
\item{1.} {\tt if} $\bigl(f(x_0) \approx f(x_1)\bigr)$ {\tt then}
\item{2.} \codent {\tt return} $\bigl( 0 = Df(x_0) = Df(x_1)
  = D^2f(x_0) = D^2f(x_1) \bigr)$
\item{3.} {\tt endif}
\item{4.} {\tt if} $\bigl(Df(x_0) < 0$ {\tt or} $Df(x_1) < 0\bigr)$ {\tt
  then return FALSE endif}
\item{5.} $w \: x_1 - x_0$
\item{6.} $z \: f(x_1) - f(x_0)$

\item{} {The necessity of Steps 1--4 follows directly from the
  fact that $f$ is $C^2$. The following Steps 7--13 coincide with a
  simplified condition for quintic monotonicity that reduces to one of
  cubic positivity studied by Schmidt and He{\ss} [1988]. Given
  $\alpha$, $\beta$, $\gamma$, and $\delta$ as defined by Schmidt and
  He{\ss}, monotonicity results when $\alpha \geq 0$, $\delta \geq 0$,
  $\beta \geq \alpha - 2 \sqrt{\alpha \delta}$, and $\gamma \geq
  \delta - 2 \sqrt{\alpha \delta}$.  Step 4 checked for $\delta < 0$,
  Step 8 checks $\alpha < 0$, Step 10 checks $\beta < \alpha - 2
  \sqrt{\alpha \delta}$, and Step 11 checks $\gamma < \delta - 2
  \sqrt{\alpha \delta}$. If none of the monotonicity conditions are
  violated, then the \red{degree five} piece is monotone and Step 12
  concludes.}

\item{7.} {\tt if} $\bigl(Df(x_0) \approx 0$ {\tt or} $Df(x_1) \approx
0\bigr)$ {\tt then}
\item{8.} \codent {\tt if} $\bigl(D^2f(x_1)w > 4Df(x_1)$ {\tt then
return FALSE endif}
\item{9.} \codent $t \: 2 \sqrt{Df(x_0) (4Df(x_1) - D^2f(x_1) w) }$
\item{10.} \codent {\tt if} $\bigl(t + 3Df(x_0) + D^2f(x_0)w < 0 \bigr)$
  {\tt then return FALSE endif}
\item{11.} \codent {\tt if} $\bigl(60z - w\bigl(24Df(x_0) + 32Df(x_1) - 2t
  + w(3D^2f(x_0) - 5D^2f(x_1))\bigr) < 0\bigr)$
\item{}   \codent \codent {\tt then return FALSE endif}
\item{12.} \codent {\tt return TRUE}
\item{13.} {\tt endif}

\item{} {The following code considers the full quintic monotonicity
case studied by Ulrich and Watson [1994]. Given $\tau_1$, $\alpha$,
$\beta$, and $\gamma$ as defined by Ulrich and Watson, a quintic
piece is proven to be monotone if and only if
$\tau_1 > 0$, and $\alpha, \gamma > -(\beta+2)/2$ when $\beta \leq 6$,
and $\alpha, \gamma > -2 \sqrt{\beta-2}$ when $\beta > 6$.
Step 14 checks $\tau_1 \le 0$, Steps 19 and 20 determine monotonicity based
on $\alpha$, $\beta$, and $\gamma$.}

\item{14.} {\tt if} $\bigl( w\bigl(2\sqrt{Df(x_0)\,Df(x_1)} - 3(Df(x_0) +
  Df(x_1))\bigr) - 24z \leq 0 \bigr)$
\item{} \codent {\tt then return FALSE endif}
\item{15.} $t \: \bigl(Df(x_0)\, Df(x_1)\bigr)^{3/4}$
\item{16.} $\alpha \: (4 Df(x_1) - D^2f(x_1)w) \sqrt{Df(x_0)} / t$
\item{17.} $\gamma \: (4 Df(x_0) - D^2f(x_0)w) \sqrt{Df(x_1)} / t$
\item{18.} $\displaystyle \beta \: {60z/w + 3\bigl(w(D^2f(x_1) -
  D^2f(x_0)) - 8(Df(x_0) + Df(x_1))\bigr) \over 2 \sqrt{Df(x_0)\,Df(x_1)}}$
\item{19.} {\tt if} $(\beta \leq 6)$ {\tt then return}
$\bigl( \min\{\alpha,\gamma\} > - (\beta + 2) / 2 \bigr)$
\item{20.} {\tt else} {\tt return}
$\bigl( \min\{\alpha,\gamma\} > -2 \sqrt{\beta - 2}\,\bigr)$
\item{21.} {\tt endif}
}
\vskip 5mm
%% ----------------------------------------------------------------------

It is shown by Ulrich and Watson [1994] that when $0 = DQ(X_i) =
DQ(X_{i+1}) = D^2Q(X_i) = D^2Q(X_{i+1})$, the quintic polynomial
over $[X_i$, $X_{i+1}]$ is guaranteed to be monotone. Using this fact, the
following algorithm shrinks (in magnitude) initial derivative estimates
until a monotone spline is achieved and outlines the core routine in the
accompanying package.

%% ===================================================================
%%       Algorithm 3: MQSI - Monotone quintic spline interpolant
\vskip 3mm
{\parindent 0mm
{\bf Algorithm 3:}
{\tt MQSI}$\bigl(X(1{:}n), Y(1{:}n) \bigr)$

\nobreak
where $(X_i,Y_i) \in \bbb{R}\times\bbb{R}$, $i = 1$, $\ldots$, $n$ are data
points. Returns monotone quintic spline interpolant $Q(x)$ such that
$Q(X_i) = Y_i$ and is monotone increasing (decreasing) on all
intervals that $Y_i$ is monotone increasing (decreasing).

}
{\parindent=6mm
\item{} Approximate first and second derivatives at $X_i$ with {\tt
  QUADRATIC\_FACET}.
\item{} {\tt for} $i\:1$ {\tt step} 1 {\tt until} $n$ {\tt do}
\item{} \codent $(u_i$, $v_i)\:$ {\tt QUADRATIC\_FACET}$(X$, $Y$, $i)$
\item{} {\tt enddo}
\item{} Identify and store all intervals where $Q$ is nonmonotone in a
queue $\cal Q$.
\item{} {\tt for} $i\:1$ {\tt step} 1 {\tt until} $n-1$ {\tt do}
\item{} \codent {\tt if not IS\_MONOTONE}$(X_i$, $X_{i+1}$, $Y_i$, $u_i$,
$v_i$, $Y_{i+1}$, $u_{i+1}$, $v_{i+1})$ {\tt then}
\item{} \codent\codent Add interval $\bigl(X_i$, $X_{i+1}\bigr)$ to
queue $\cal Q$.
\item{} \codent {\tt endif}
\item{} {\tt enddo}
\item{} {\tt do while} $\bigl($ queue $\cal Q$ of intervals is nonempty $\bigr)$
\itemitem{} Shrink (in magnitude) $DQ$ (in $u$) and $D^2Q$ (in $v$)
  that border intervals where $Q$ is nonmonotone.  
\itemitem{} Identify and store remaining intervals where $Q$ is
  nonmonotone in queue $\cal Q$.
\item{} {\tt enddo}
\item{} Construct and return a $B$-spline representation of $Q(x)$.
}
\vskip 3mm
%% ----------------------------------------------------------------------

Since {\tt IS\_MONOTONE} can handle both nondecreasing and
nonincreasing simultaneously by taking into account the sign of $z$,
Algorithm 3 produces $Q(x)$ that is monotone increasing (decreasing)
over exactly the same intervals that the data $(X_i$, $Y_i)$ is
monotone increasing (decreasing).

Given the minimum \red{curvature (minimum magnitude of the second derivative)}
nature of the initial derivative estimates, it is desirable to make
the smallest necessary changes to the initial interpolating spline $Q$
while enforcing monotonicity. In practice a binary search for the
boundary of monotonicity is used in place of solely shrinking $DQ$ and
$D^2Q$ at breakpoints adjoining {\it active\/} intervals: intervals
over which $Q$ is nonmonotone at least once during the search. The
binary search considers a Boolean function $B_i(s)$, for $0 \le s \le
1$, that is true if the order six polynomial piece of $Q(x)$ on
$[X_i, X_{i+1}]$ matching derivatives
$Q(X_i)=Y_i$, $DQ(X_i)=s \,u_i$, $D^2Q(X_i)=s \,v_i$ at $X_i$, and 
derivatives $Q(X_{i+1})=Y_{i+1}$, $DQ(X_{i+1})=s \,u_{i+1}$, 
$D^2Q(X_{i+1})=s \,v_{i+1}$ at $X_{i+1}$ is monotone, and false
otherwise.  The binary search is only applied at those breakpoints
adjoining intervals $[X_i, X_{i+1}]$ over which $Q$ is nonmonotone and
hence $B_i(1)$ is false.  It is further assumed that there exists
$0 \le s^* \le 1$ such that $B_i(s)$ is true for $0 \le s \le s^*$ and
false for some $1 > s > s^*$. Since the derivative conditions at
interior breakpoints are shared by intervals left and right of the
breakpoint, the binary search is performed at all breakpoints
simultaneously.  Specifically, the monotonicity of $Q$ is checked on
all active intervals in each step of the binary search to determine
the next derivative modification at each breakpoint. The goal of this
search is to converge on the boundary of the monotone region in the
$(\tau_1$, $\alpha$, $\beta$, $\gamma)$ space (described in Ulrich and
Watson [1994]) for all intervals. This multiple-interval binary search
allows the value zero to be obtained for all (first and second)
derivative values in a fixed maximum number of computations, hence has
no effect on computational complexity order. This binary search
algorithm is outlined below.


%% ===================================================================
%%              Algorithm 4: Global binary search
\vskip 5mm {\parindent 0mm {\bf Algorithm 4:}
{\tt BINARY\_SEARCH}$\bigl(X(1{:}n), Y(1{:}n), u(1{:}n), v(1{:}n) \bigr)$

\nobreak
where $(X_i,Y_i) \in \bbb{R}\times\bbb{R}$, $i = 1$, $\ldots$, $n$ are
data points, and $Q(x)$ is a quintic spline interpolant such that
$Q(X_i) = Y_i$, $DQ(X_i) = u_i$, $D^2Q(X_i) = v_i$. Modifies
derivative values ($u$ and $v$) of $Q$ at data points to ensure
{\tt IS\_MONOTONE} is true for all intervals defined by adjacent data
points, given a desired precision $\mu \in {\bbb R}$. proj$\bigl(w$,
int$(a,b)\bigr)$ denotes the projection of $w$ onto the closed
interval with endpoints $a$ and $b$.

}
{\parindent=6mm
\item{} Initialize the step size $s$, make a copy of data defining $Q$,
and construct three queues necessary for the multiple-interval binary search.
\item{} $s \: 1$
\item{} $(\hat u,\hat v) \: (u,v)$
\item{} {\tt searching} $\:$ {\tt TRUE}
\item{} {\tt checking} $\:$ empty queue for holding left indices of {\it intervals}
\item{} {\tt growing} $\:$ empty queue for holding indices of {\it data points}
\item{} {\tt shrinking} $\:$ empty queue for holding indices of {\it data points}
\item{} {\tt for} $i\:1$ {\tt step} 1 {\tt until} $n-1$ {\tt do}
\itemitem{} {\tt if not IS\_MONOTONE}$\bigl( X_i$, $X_{i+1}$, $Y_i$, $u_i$,
$v_i$, $Y_{i+1}$, $u_{i+1}$, $v_{i+1} \bigr)$ {\tt then}
\itemitem{} \codent Add data indices $i$ and $i+1$ to queue {\tt shrinking}.
\itemitem{} {\tt endif}
\item{} {\tt enddo}
\item{} {\tt do while} $\bigl(${\tt searching or} ({\tt shrinking} is
nonempty)$\bigr)$
\itemitem{} Compute the {\it step size} $s$ for this iteration of the search.
\itemitem{} {\tt if  searching  then} $s \: \max\{\mu, s/2\}$
{\tt else} $s \: 3s/2$ {\tt endif}
\itemitem{} {\tt if} $(s=\mu)$ {\tt then searching} $\:$ {\tt FALSE};
  clear queue {\tt growing endif}
\itemitem{} Increase in magnitude $u_i$ and $v_i$ for all data indices
  $i$ in {\tt growing} such that the points $X_i$ are strictly adjoining
  intervals over which $Q$ is monotone.
\itemitem{} {\tt for} ($i \in$ {\tt growing}) {\tt and} ($i \not\in$
{\tt shrinking}) {\tt do}
\itemitem{} \codent $u_i \: \hbox{proj}\bigl(u_i + s\,\hat u_i, \hbox{int}
  (0,\hat u_i)\bigr)$
\itemitem{} \codent $v_i \: \hbox{proj}\bigl(v_i + s\,\hat v_i, \hbox{int}
  (0,\hat v_i)\bigr)$
\itemitem{} \codent Add data indices $i-1$ (if not 0) and $i$ (if not $n$) to queue
  {\tt checking}.
\itemitem{} {\tt enddo}
\itemitem{} Decrease in magnitude $u_i$ and $v_i$ for all data indices
  $i$ in {\tt shrinking} and ensure those data point indices are placed
  into {\tt growing} when {\tt searching}.
\itemitem{} {\tt for} $i \in$ {\tt shrinking do}
\itemitem{} \codent If {\tt searching}, then add index $i$ to queue
  {\tt growing} if not already present.
\itemitem{} \codent $u_i \: \hbox{proj}\bigl(u_i - s\,\hat u_i, \hbox{int}
  (0,\hat u_i)\bigr)$
\itemitem{} \codent $v_i \: \hbox{proj}\bigl(v_i - s\,\hat v_i, \hbox{int}
  (0,\hat v_i)\bigr)$
\itemitem{} \codent Add data indices $i-1$ (if not 0) and $i$ (if not $n$) to queue
  {\tt checking}.
\itemitem{} {\tt enddo}
\itemitem{} Empty queue {\tt shrinking}, then check all intervals left-indexed
  in queue {\tt checking} for monotonicity with {\tt IS\_MONOTONE},
  placing data endpoint indices of intervals over which $Q$ is nonmonotone
  into queue {\tt shrinking}.
\itemitem{} Clear queue {\tt shrinking}.
\itemitem{} {\tt for} $i \in$ {\tt checking do}
\itemitem{} \codent
  {\tt if not IS\_MONOTONE}$\bigl( X_i$, $X_{i+1}$, $Y_i$, $u_i$,
  $v_i$, $Y_{i+1}$, $u_{i+1}$, $v_{i+1} \bigr)$ {\tt then}
\itemitem{} \codent \codent Add data indices $i$ and $i+1$ to {\tt shrinking}.
\itemitem{} \codent {\tt endif}
\itemitem{} {\tt enddo}
\itemitem{} Clear queue {\tt checking}.
\item{} {\tt enddo }
}
\vskip 5mm
%% ----------------------------------------------------------------------

%% \itemitem{} Take search steps either modifying derivative values
%%   towards monotonicity (zero) or towards the original values over all
%%   intervals that are either nonmonotone or previously modified.


In the subroutine {\tt MQSI}, $\mu = 2^{-26}$, which results in 26
guaranteed search steps for all intervals that are initially
nonmonotone. An additional 43 steps could be required to reduce a
derivative magnitude to zero with step size growth rate of $3/2$. This
can only happen when $Q$ becomes nonmonotone on an interval for the
first time while the step size equals $\mu$, but for which the only
viable solution is a derivative value of zero. The maximum number of
steps is due to the fact that $\sum_{i=0}^{42} \mu (3/2)^i > 1$. In
total {\tt BINARY\_SEARCH} search could require 69 steps.
