\heading{4. COMPLEXITY AND SENSITIVITY}

Algorithms 1 and 4 have ${\cal O}(n)$ runtime for $n$ data
points. Algorithm 2 has a fixed cost ${\cal O}(1)$. Given a fixed
schedule for shrinking derivative values, Algorithm 3 has a ${\cal
  O}(n)$ runtime for $n$ data points. In execution, the majority of
the time, still ${\cal O}(n)$, is spent solving the banded linear
system of equations for the B-spline coefficients. Thus for $n$ data
points, the overall execution time is ${\cal O}(n)$.

The quadratic facet model produces a unique sensitivity to input
perturbation, as small changes in input may cause different quadratic
facets to be associated with a breakpoint, and thus different initial
derivative estimates can be produced. This phenomenon is depicted in
Figure 1. Despite this sensitivity, the quadratic facet model is still
preferred because it exactly captures local linear and quadratic
behavior while empirically producing final approximations with less
wiggle (local $L^2$ norm of the second derivative) than other methods. A
weighted harmonic mean estimate of first derivatives may be more
accurate when the underlying function changes at a rate greater than a
quadratic, but that method increases the second derivative sensitivity
to small perturbations in data and empirically results in quintic
splines with greater wiggle.

The binary search for a point on the monotone boundary in $(\tau_1$,
$\alpha$, $\beta$, $\gamma)$ space is performed because it results in
monotone quintic spline interpolants with derivative values that are
absolutely nearer to initial estimates than a search that strictly
shrinks derivative values. Given that the initial derivative estimates
have desirable properties (capture low-order phenomena and are low
wiggle), this search results in an approximation that is both monotone
and has derivative values similar to the initial estimates.


\topinsert
\centerline{\epsfxsize=4truein \epsffile{vis/1-sensitivity.eps}}
{\everymath={\scriptstyle}

\narrower\noindent\rmVIII Fig.\ 1. A demonstration of the quadratic
  facet model's sensitivity to small data perturbations. This example is
  composed of two quadratic functions $f_1(x) = x^2$ over points $\{1$,
  $2$, $5/2\}$, and $f_2(x) = (x-2)^2 + 6$ over points $\{5/2$, $3$,
  $4\}$. Notably, $f_1(5/2) = f_2(5/2)$ and $f_1$, $f_2$ have the same
  curvature (equal second derivatives). Given the exact five data points
  seen above, the quadratic facet model produces the slope seen in the
  solid blue line at $x = 5/2$. However, by subtracting the value of
  $f_3$ $= \epsilon(x-2)^2$ from points at $x = 3$, 4, where $\epsilon$
  is the machine precision ($2^{-52}$ for an IEEE 64-bit real), the
  quadratic facet model produces the slope seen in the dashed red line
  at $x = 5/2$. This is the nature of a facet model and a side effect
  of associating data with local facets.
\par} \endinsert
